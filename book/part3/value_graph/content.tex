
\chapter*{Program Analyses and Optimizations on the Value Graph: Constant Propagation and More}

\chapterauthor{Jens Knoop, Oliver R\"uthing}

\section{Abstract}

The value graph melts the data flow and control flow of a program in
SSA form into a single data structure. This data structure has been
invented by Alpern, Wegman, and Zadeck [AWZ-POPL88], and originally
been used to detect variables of equal value in programs. Since then
the value graph proved most useful for other program analyses and
program optimizations, too. These applications are distinct for their
conceptual simplicity and elegance, which is owed to the unique
representation of data and control flow in the SSA form of a program
and condensed further in the value graph. These applications are thus
particularly suited to demonstrate the benefits of SSA form and the
value graph derived from it for program analysis and optimization.

In this section of the book we provide evidence for this by
highlighting a representative sample of these
applications. Essentially, these applications fall into two
groups. Applications of the first group operate directly on the value
graph, e.g. by means of graph transformations. Here we start from the
original application of Alpern, Wegman, and Zadeck for variable
equality detection [AWZ-POPL88], and subsequently investigate how to
enhance the precision of this algorithm by integrating additional
graph transformations [RKS-SAS99]. Applications of the second group
use the value graph as a sparse structure for performing conventional
data flow analysis. Here we present constant propagation on the value
graph as a model application [KR-CC00,KR-JUCS03] and its extension to
predicated SSA form [KR-SBLC03]. Depending on space available, we
conclude the section surveying other applications.



\section{Page Estimate}

- Representing the technical core of the section (i.e., variable
 equivalence detection plus constant propagation on the value graph):
 ca. 15 pages
- Surveying further applications: 10 to 20 pages (by appropriate 
 selection of surveyed techniques and granularity level of presentation
 the length of this part can be adjusted more flexibly.


% References
% ----------
% 
% [AWZ-POPL88]
%   Alpern, B., Wegman, M. N., and Zadeck, F. K.
%   Detecting Equality of Variables in Programs.
%   In Conf. Rec. 15th Symp. Principles of Programming
%   Languages and Systems (POPL'88), ACM, NY,
%   1988, 1 - 11.
% 
% [KR-CC00] 
%   Knoop, J., and Rüthing, O. 
%   Constant Propagation on the Value Graph:
%   Simple Constants and Beyond. In Proc. 9th
%   Int. Conf. on Compiler Construction (CC 2000), 
%   LNCS 1781 (2000), 94 - 109.
% 
% [KR-JUCS03]
%   Knoop, J., and Rüthing, O.
%   Constant Propagation on Predicated Code.
%   J. of Universal Computer Science 9, 8 (2003), 
%   829 - 850. (special issue devoted to SBLP'03).
% 
% [KR-SBLC03]
%   Knoop, J., and Rüthing, O.
%   Constant Propagation on Predicated Code. In 
%   Proc. 7th Brazilian Symp. on Programming Languages 
%   (SBLP 2003), 135 - 148.
% 
% [RKS-SAS99]
%   Rüthing, O., Knoop, J., and Steffen, B. 
%   Detecting equality of variables: Combining efficiency with precision.
%   In Proc. 6th International Static Analysis Symposium (SAS'99), 
%   LNCS 1694 (1999), 232 - 247.
