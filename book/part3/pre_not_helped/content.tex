\chapter{PRE \Author{F. Chow, J. Knoop and O. R\"uthing}}
\numberofpages{17}
\chapterauthor{J. Knoop and O. R\"uthing}

\section{Introduction}
{\bf 3 pages}
\begin{itemize} 
\item Introduction to partial redundancy elimination.
\item A brief overview on related work and history.
\item Sketch of the Lazy Code Motion (LCM) algorithm.
\end{itemize}

\section*{The Basic Algorithm - SSAPRE}
{\bf 5 pages} \\
Present Kennedy et al.'s the  SSA-adaption  of LCM. 

\section*{Register Promotion via PRE}
{\bf 2 pages} \\
Present register promotion as an application that can be 
can be treated by techniques dual to PRE. Discuss its 
relationship to Partial Dead Code Elimination. This section 
will be mainly based on the algorithm of Lo, Chow et al. 
 
\section*{Speculative  PRE}
{\bf 3 pages} \\
Discuss  extensions of SSAPRE incorporating speculative code 
motion. This requires to weaken the notion of down-safety. 
Present Xu and Cai's algorithm for speculative PRE based on 
edge profiles,  and  Murphy et al.'s algorithm for code motion 
of fault-safe instructions. 

\section*{Semantic PRE}
{\bf 4 pages} \\
Present some approaches to PRE incorporating information from global 
value numbering. In particular, discuss the  algorithm of Rosen, Wegman and 
 Zadeck and the algorithm of Click and Cooper. 





%\TODO{as a counter example}


