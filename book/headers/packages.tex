\RequirePackage{ifthen}

% use only scalable T1 fonts (no T3 please!)
\usepackage[T1]{fontenc}
\usepackage{lmodern}

\usepackage{cancel}
\usepackage{etex}
\usepackage{index}
  \proofmodetrue % comment this to make index entries disappear from the top corner pages
  \indexproofstyle{\footnotesize\it}

% show index notes directly at their position
\ifproofmode
  \let\oldindex=\index
  \def\index#1{\tikzshowindex{#1}\oldindex{#1}}
\fi
% main def
\newcommand{\mainidx}[1]{\underline{#1}}
\newcommand{\exampleidx}[1]{\textit{#1}}
\newcommand{\seealso}[2]{{\it See also} #1}


\usepackage{marginnote}
\usepackage{fancyhdr}
\setlength\headheight{12.05pt}
\usepackage{titlesec} % allows to change style of chapters, sections, etc.
\usepackage{paralist}

\usepackage{verbatim} % better handling of verbatim than original LaTeX
% local packages
\ssausepackage{progressbar}  % to have ``progress bars'' for each chapter
%Algorithms
\usepackage[ruled, vlined, linesnumbered, algochapter]{algorithm2e}  
% algorithm2e.pdf in the book/packages directory
% \ssausepackage[ruled, vlined, linesnumbered, algochapter]{algorithm2e}  % algorithm2e.pdf in the book/packages directory

\let\oldnl\nl% Store \nl in \oldnl
\newcommand{\nonl}{\renewcommand{\nl}{\let\nl\oldnl}}% Remove line number for one line

\ssausepackage{evenoddcheck} % to check two parts are visible at the same time

%\usepackage[pdftex]{graphicx}
\usepackage{graphicx}
\DeclareGraphicsRule{.pdftex}{pdf}{.pdftex}{}
\usepackage{floatrow}
\usepackage{subfig}   % replace the older ``subfigure'' package by same author
%\usepackage{caption}
\usepackage{wrapfig}  % wrap text around figures
\usepackage{shapepar} % Write paragraph in any shape

% automatically loaded by subfig, but put there no know that captions are all
% modified and to put back Springer's svmono style.

\usepackage{datetime}

% Put the time in hh:mm format.
\settimeformat{xxivtime}

% Write latex source info (file, line) in dvi file
\usepackage{srcltx}

% Simplify handling of spaces after macros
\usepackage{xspace}

% Different colors to ease writing of comments & todos in text
\usepackage{xcolor}

% To comment large parts of text using \begin{comment}...\end{comment}
\usepackage{comment}

\usepackage{listings}
\lstset{language=C}

%\newcommand{\bmmax}{0}
%\newcommand{\hmmax}{0}
\usepackage{alltt}
\usepackage{amsmath, stmaryrd}
%% removed many math packages or it will use too many math alphabets for tex!
%\usepackage{amsmath, amsfonts, amssymb, amsxtra, amsopn, txfonts}
%\usepackage{mathtools} % to use \mathclap & other maths equivalents of \rlap, etc.
%\usepackage{MnSymbol}
\usepackage[adobe-utopia]{mathdesign}
\usepackage{enumerate}
\usepackage{multicol}

%\usepackage{tikz}
%\usetikzlibrary{shapes,arrows,calc}

\usepackage{url}

% For hyperlinks in pdf
\usepackage[%pdftex, %draft,
  pdftitle={SSA-based Compiler Design},
  pdfauthor={Fabrice Rastello et al.},
  pdfsubject={Compiler Book},
  pdfkeywords={Compilers, Static Single Assignment},
  final,
  hyperindex=true,
  plainpages=false, pdfpagelabels, hypertexnames=false, pagebackref,
  % pdfview=Fit,  % << Never use this option or it will apply to all links (and screw them...)
  bookmarks, bookmarksopen, bookmarksnumbered=true, 
  %
  colorlinks=true,linkcolor=red!30!black, citecolor=blue!50!black, 
  urlcolor=blue!40!red,
  pdfborder={0 0 0},
  %
  bookmarksopenlevel=1,
  breaklinks,
  % naturalnames was required for old versions of algorithm2e, not anymore
]{hyperref}


