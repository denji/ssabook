% Set the input path for a chapter
% Usage: \inputpath{partX}{chapY}
%    or: \inputpath[figs]{partX}{chapY} if figures are not in img/

\makeatletter
\newcommand\inputpath[3][img]{
  \def\input@path{{#3/}{#2/#3/}}
  \graphicspath{{#1/}%
                {#3/#1/}%
                {#2/#3/#1/}%
                {#3/}%
                {#2/#3/}%
                }
}
\makeatother


% tempbox for floats alignments
\newsavebox{\tempflbox}%
\newcommand\defineheight[1]{%
  \sbox{\tempflbox}{#1}%
}
\newcommand\useheightbox{%
  \usebox{\tempflbox}%
}
\newcommand\defuseheight[1]{%
  \defineheight{#1}%
  \usebox{\tempflbox}%
}
\newcommand\centerheight[1]{%
    \vbox to \ht\tempflbox{%
    \vfil%
    \hbox{#1}%
    \vfil%
  }%
}



% Not sure what to do with these yet.
\newcommand{\chapterauthor}[1]{{\textbf{Author: #1}}}
\newcommand{\parteditor}[1]{{\textbf{Editor: #1}}}



%% Defines some environments




% Simple commands
%\newcommand{\TODO}[1]{{\textbf{TODO: #1}\\}}
%
\ifx\dousetikz\undefinedmacro
  \newcommand{\todo}[1]{\marginpar{\footnotesize\color{red!50!black}TODO: #1}}
\else
  \usepackage[textsize=scriptsize]{todonotes}
\fi
\let\TODO\todo

\newcommand{\warning}[1]{{\footnotesize\color{red!50!black}Warning: #1}}


% internal macro for chapter pages
\gdef\mychappagenum{0}

% Number of pages. Thanks to Florent Bouchez for this trick
\newcommand\numberofpages[1]{
  {% new group to make changes to \count0 local
  \ifnum\mychappagenum>0
   \applynumberofpages
  \fi
  \count0=\value{page}
  \advance\count0 by #1
  \xdef\mychappagenum{\the\count0}
  \message{^^JChapter should end at page \mychappagenum^^J}
  }
}

% internal command for chapter pages
%\newcommand\applynumberofpages{{
%  \ifnum\mychappagenum>0
%   \ifnum\value{page}>\mychappagenum
%   \count0=\value{page}
%%    \errmessage{Error: chapter should end before page {\mychappagenum}  but page
%%    counter is already at {\the\count0}.}
%   \else
%      \setcounter{page}{\mychappagenum}
%      \xdef\mychappagenum{0}
%   \fi
% \fi
% }}

% authors name in TOC
\newcommand\Author[1]{--- (\emph{#1})}
\newcommand\andAuthor{\unskip, } % separate authors by \andAuthor in every chapter
% no newline in TOC
\newcommand\chapternewline{}

%%%% /!\ FIX made by Fabrice to be able to compile  %%%%%%%%
\def\st{\textrm{s.t.}}
\def\J{{\cal J}}
\def\minus{\backslash}

%%%% Macros used by Das/Ramakrishna in
%%%% part1/alternative_ssa_construction_algorithms/

%\newcommand {\edge}[1] {\stackrel{#1}{\rightarrow}}
\newcommand{\edge}[1]{\;\ooalign{$\rightarrow$\cr\hfil\raisebox{.8ex}{\footnotesize $#1$}\hfil}\;}

\newlength{\identamt}
\setlength{\identamt}{1.0em}

\newcounter{linectr}
\def\x#1{\refstepcounter{linectr}\item\hspace*{#1\identamt}}
\def\xx#{\hspace*{\identamt}}

\newenvironment{code}{\begin{list}{\bf \arabic{linectr}:\hfill} %
{\itemsep=0pt \topsep=0pt \parsep=0pt \leftmargin=\labelwidth \labelsep=0pt}}%
{\end{list}}

%%%% end of Das/Ramakrishna macros

%%%% Proposed macros by F. Bouchez to unify notations
%%%% used in part4/register_allocation


%%%% end of F. Bouchez macros

%%% Used by code selection and array ssa chapter
\newcommand{\eg}{e.g.,\xspace}
\newcommand{\ie}{i.e.,\xspace}
%%%

%%% Fabrice's macros
\newcommand{\GVN}[1]{\textrm{H}(#1)}
\newcommand{\join}{{\cal J}}
\newcommand{\Split}{{\cal S}}
\renewcommand{\DJ}{\textrm{DJ}\xspace}
\newcommand{\DF}{\textrm{DF}\xspace}
\newcommand{\iDF}{\mbox{$\textrm{DF}^+$}\xspace}
\newcommand{\pDF}{\textrm{pDF}\xspace}
\newcommand{\valundef}{\bot}
\newcommand{\ipDF}{\mbox{$\textrm{pDF}^+$}\xspace}
\newcommand{\bottom}{\bot}
\newcommand\cond{~?~}
\newcommand{\nonnull}{\xcancel{0}}
\newcommand\var[1]{\textit{#1}}
\newcommand{\web[1]}{\textrm{SSAweb}(#1)}
\newcommand{\Defs}[1]{D_{#1}}
\newcommand{\defsv}{\Defs{v}}
\newcommand\early[1]{\textsf{early}(#1)}
\newcommand\late[1]{\textsf{late}(#1)}
\newcommand\level{\textsf{level}\xspace}
\newcommand\phifun{$\phi$-function\xspace}
\newcommand\phifuns{$\phi$-functions\xspace}
\newcommand\PHIfun{$\Phi$-function\xspace}
\newcommand\PHIfuns{$\Phi$-functions\xspace}
\newcommand\psifun{$\psi$-function\xspace}
\newcommand\psifuns{$\psi$-functions\xspace}
\newcommand\phinode{$\phi$-node\xspace}
\newcommand\phinodes{$\phi$-nodes\xspace}
\newcommand\PHInodes{$\Phi$-nodes\xspace}
\newcommand\phiuse{$\phi$-use\xspace}
\newcommand\PHIuse{$\Phi$-use\xspace}
\newcommand\PHIuses{$\Phi$-uses\xspace}
\newcommand\phiuses{$\phi$-uses\xspace}
\newcommand\phidef{$\phi$-definition operand\xspace}
\newcommand\PHIdef{$\Phi$-definition operand\xspace}
\newcommand\phiop{$\phi$-operator\xspace}
\newcommand\phiops{$\phi$-operators\xspace}
\newcommand\useop{use-operand\xspace}
\newcommand\useops{use-operands\xspace}
\newcommand\defop{definition-operand\xspace}
\newcommand\sigmafun{$\sigma$-function\xspace}
\newcommand\sigmafuns{$\sigma$-functions\xspace}
\newcommand\chifun{$\chi$-function\xspace}
\newcommand\chifuns{$\chi$-functions\xspace}
\newcommand\mufun{$\mu$-function\xspace}
\newcommand\mufuns{$\mu$-functions\xspace}
\newcommand\gammafun{$\mu$-function\xspace}
\newcommand\gammafuns{$\mu$-functions\xspace}
\newcommand\chiop{$\chi$-operator\xspace}
\newcommand\chiops{$\chi$-operators\xspace}
\newcommand\muop{$\mu$-operator\xspace}
\newcommand\muops{$\mu$-operators\xspace}
\def\downsafe{\textit{downsafe}\xspace}
\def\hasrealuse{\textit{has\_real\_use}\xspace}
\def\canbeavail{\textit{can\_be\_avail}\xspace}
\def\willbeavail{\textit{will\_be\_avail}\xspace}
\def\tauvar{$\tau$ variable\xspace}
\def\tauvars{$\tau$ variables\xspace}
\def\tauop{$\tau$ operand\xspace}
\def\tauops{$\tau$ operands\xspace}
\def\tauedge{$\tau$-edge\xspace}
\def\later{\textit{later}\xspace}
\def\avail{\textit{avail}\xspace}
\def\gets{\leftarrow}
\def\phiif{\phi_\textit{if}}
\def\phientry{\phi_\textit{entry}}
\def\phiexit{\phi_\textit{exit}}
\def\phiweb{$\phi$-web\xspace}
\def\phiwebs{$\phi$-webs\xspace}
\def\phipsiweb{$\psi$-$\phi$-web\xspace}
\def\phipsiwebs{$\psi$-$\phi$-webs\xspace}
\newcommand\proba[1]{\textsf{prob}(#1)}
\newcommand\cost[1]{\widehat{#1}}
\def\brlat{\textsf{br\_lat}}
\def\psiweb{$\psi$-web\xspace}
\def\psiwebs{$\psi$-webs\xspace}
\newcommand{\pining}[1]{^{\uparrow #1}}
\def\SP{\textit{SP}}

\newboolean{hab}
\setboolean{hab}{false} % please don't remove. Fab's internal usage

\newcommand\false{\textnormal{False}\xspace}
\newcommand\true{\textnormal{True}\xspace}
\newcommand\nullprog{\textnormal{Null}\xspace}

\newcommand\cidom[1]{#1.\textnormal{idom}}
\newcommand\eanc[1]{#1.\textnormal{eanc}}
\def\curanc{\textit{cur\_anc}}
\def\curidom{\textsf{cur\_idom}}
\def\dominates{\textsf{dominates}\xspace}
\newcommand\dominated[1]{\textsf{dominated}(#1)}
\def\dom{\textrm{ dom }}
\def\sdom{\textrm{ sdom }}
\def\uncolored{\textsf{uncolored}}
\def\uncolor{\textsf{uncolor}}
\def\colored{\textsf{colored}}
\def\intersect{\textsf{intersect}}
\def\interfere{\textsf{interfere}}
\def\colors{\textsf{COLORS}}
\def\col{\textsf{color}}
\newcommand{\atomic}[1]{\textsf{atomic-merged-set}(#1)}
\def\curphi{\textit{curphi}}
\def\curpred{\textit{cur\_pred}}

\def\iftt{\texttt{if }}
\def\thentt{\texttt{then }}
\def\elsett{\texttt{else }}
\def\endtt{\texttt{end }}
\def\whilett{\texttt{while }}
\def\returntt{\texttt{ret }}

\def\Defs#1{\textnormal{Defs}(#1)}
\def\Def#1{\textnormal{Def}(#1)}
%%%

\makeatletter
%\def\vardef@aux#1{#1 \gets \ldots}
%\newcommand\vardef[1]{\ifmmode\vardef@aux{#1}\else$\vardef@aux{#1}$\fi}
\newcommand\vardef[1]{\ensuremath{#1 \gets \ldots}}
\def\varuse@aux#1{\ldots \gets #1}
\newcommand\varuse[1]{\ifmmode\varuse@aux{#1}\else$\varuse@aux{#1}$\fi}
\newcommand\somecode{\mbox{\kern1em\vdots}}
\makeatother

%% algorithms
\newcommand\KwInPhantom{\strut\hphantom{\bf Input:}}
