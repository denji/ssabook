\chapter{The interaction of instruction scheduling and SSA form \Author{A. Ertl}}
\numberofpages{1}
%\section{TODO}
%
%\begin{itemize}
%	\item ssa has no ordering
%	\item phase ordering
%	\item sequencing
%\end{itemize}

Instruction scheduling orders or reorders instructions; the typical
goal in instruction scheduling is to exploit instruction-level
parallelism, i.e., to arrange the instructions such that there is no
dependence between instructions that are close together.  Another
(conflicting) goal is to reduce register pressure.

The input to (basic-block) instruction scheduling is a
partially-ordered dependence graph of the instructions; the output is
a totally-ordered sequence of instructions.

The interaction of instruction scheduling with SSA form is that
instruction selection provides the instructions as a data-dependence
graph, so instruction scheduling does not have to extract the
data-dependence graph from some sequential representation (which was
discussed at length in early papers on instruction scheduling
\cite{...}).

Actually SSA form itself covers only data flow dependences, but the
other dependences (write-after-read (anti-) dependences and
write-after-write (output) dependences) have to be represented as well
to allow correct transformations (not just for instruction
scheduling).

Also, SSA-based register allocation (see Section \ref{sect:ra})
assumes that the code is totally ordered, resulting in interval graphs
for basic blocks (which is required for SSA-based register
allocation), so some kind of sequencing has to be performed before
register allocation in any case.  Most other register allocation
methods also assume totally-ordered code in order to be able to
determine whether two live ranges conflict or not.

Instruction scheduling proper is otherwise not affected by SSA form,
and there is a lot of work on basic block instruction scheduling
\cite{...} as well as more advanced forms such as superblock
scheduling \cite{...} (for single-entry multiple-exit regions), trace
scheduling \cite{...} (for multi-entry multi-exit sequences of basic
blocks), and software pipelining \cite{...} (for loops, usually simple
loops).  Therefore we will not try to give an overview here, but refer
the interested reader to the literature.
