% vim:spell:spelllang=en
\chapter{Register Allocation\Author{F. Bouchez, S. Hack}}

\numberofpages{15}
Page count: 15


\chapterauthor{Bouchez, Hack}

{\it
We will start by giving an overview and intuitive understanding why SSA helps 
register allocation. Then we present the sub-problems of spilling and 
coloring/coalescing, as two separate phases, under the light of SSA. Finally, 
practical issues like register constraints and critical edge splitting are 
discussed.

%% Intro by Hack
%We will start by giving an overview and intuitive understanding why SSA helps register allocation.
%Then, we present the principal architecture of an SSA-based register allocator.

%We discuss spilling, assignment, and coalescing in further detail and review the proposed techniques.
%Finally, we present the corresponding complexity results for each phase.
}


\section{Why does SSA help register allocation?}


\subsection{Classical register allocation}

\begin{itemize}
  \item based on graph coloring, linear scan
  \item spilling \& coloring are dependent
  \item heuristics are used to find a working solution
  \item avoiding spills is more important than coalescing
\end{itemize}


\subsection{Programs under SSA have chordal interference graphs}
\begin{itemize}
  \item Proof programs under SSA are chordal
  \item What difference does it make? Knowing the minimum number of colors is 
    easy: we know if spilling is mandatory or not
  \item why are the graphs chordal? View of SSA variables as subtrees of the 
    dominance tree; made possible because of \phifuns
  \item \phifuns allows colors to be ``re-arranged'' on incoming edges: this 
    is implicit shuffle code. For this to work, remember that semantics of 
    \phifun is they are \emph{parallel copies}. Hence, the SSA-variables of 
    same initial variable in different branches of dominance tree are 
    color-independent.
\end{itemize}

\subsection{Maxlive and colorability of graphs}
\begin{itemize}
  \item definition of Maxlive
  \item show that Maxlive is the minimum number of colors required
  \item scan of the dominance tree sufficient to know Maxlive: pseudo-code
\end{itemize}
    


\section{The problem of spilling}

\subsection{Lowering Maxlive}
\begin{itemize}
  \item The min number of colors is Maxlive. Variables must be stored in memory 
    to lower the register pressure at points where it is $>R$.
  \item Whenever, Maxlive $\leq R$, we know for sure that spilling more in 
    unnecessary.
\end{itemize}

\subsection{SSA does not help for spilling}
\begin{itemize}
  \item easy spill method: spill everywhere
  \item SSA split points are good for coloring, but not really for spilling 
    (see article spill everywhere under SSA)
  \item other split points to avoid spill everywhere are probably better
\end{itemize}


\subsection{Spilling under SSA}
Still, we give a possibility to perform spilling under SSA
\begin{itemize}
  \item Hack's algorithm: simplified version?
  \item \`a la Belady following dominance tree?
\end{itemize}



\section{Coloring and coalescing}
\subsection{Chordal property for coloring}
\begin{itemize}
  \item two simplicial nodes
  \item color in the reverse order of simplification
  \item one such order considers subtrees from the root of the tree
  \item no need to construct interference graph
\end{itemize}

\subsection{Fast tree-scan solution}
\begin{itemize}
  \item On the dominance tree, possible to scan from top and assign colors to 
    variables as they come.
  \item Biased coloring is used to coalesce variables linked by \phifuns.
  \item see next section for more involved coalescing
\end{itemize}



\section{Practical and advanced discussions}

\subsection{Handling registers constraints}
\begin{itemize}
  \item ABI contraints
  \item split beforehand solution: many parallel copies, need good biased 
    coloring
  \item repair afterwards solution: biased try to give right color, else, 
    actual copies are added
\end{itemize}


\subsection{Out-of-SSA and critical edge splitting}
\begin{itemize}
  \item \phifuns are not machine instructions, they are replaced by actual 
    copies in out-of-SSA phase
  \item in our case, colored SSA: Sreedhar not possible
  \item SSA implicitly 
    actual ``parallel copies'' are placed on edges
  \item problem with some critical edges: abnormal edges, back-edge of loop
\end{itemize}


\subsection{More repairing on \phifuns}
\begin{itemize}
  \item Biased coloring is fast but not very good at coalescing
  \item Basic parallel copy motion: move to next edge.
  \item More involved problem: spill variables
  \item Better solution: look at biblio (parallel copy motion)
\end{itemize}


