\chapter{Introduction \Author{S. Hack}}
\section{TODO}


Size: 10 pages

Using SSA in code generation has multiple advantages.
First, we can reuse the program representation of the ''middle end''.
If the middle end's IR is carefully designed, no additional IR has to be incorporated into the compiler.
Second, SSA is most natural for many back-end tasks.
Register allocation greatly profits from the single-assignment property.
Scheduling and instruction selection can directly use the data-dependence graphs inherent to SSA.

Furthermore, the representation of low-level details (such as registers, instructions, stack frames, and so on) is most important for any low-level IR.
We show how these can be nicely adopted in an SSA-based program representation.

\section{overview}
\section{phase ordering problem}
\section{modelling instruction effects in SSA}
