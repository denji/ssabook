\applynumberofpages\chapter{Alternative SSA construction/destruction algorithms \Author{Ramakrishna? and F. Rastello}}
\numberofpages{10}

\section{SSA construction using DJ-graphs \Author{Ramakrishna?}}

\section{SSA destruction for machine code \Author{F. Rastello}}
\label{sec:advanced_destruction}
\textbf{5 pages}
Describes the context: cannot necessarily split critical edges, some variables have renaming constraints, some instructions have renaming constraints that we want to handle during SSA destruction (some may be handled during register allocation). 

Renaming constraints are handled using copies around constrained instructions. Go to conventional-SSA using Sreedhar's~I technique. Outline the limitation of this technique when dealing with condional jump instruction that define a variable itself; when dealing with renaming constraints for variables that interfere...

Define our ``ultimate'' notion of interference using value. Build the interference graph (provide the simplified pseudo-code). Do coalescing. Refer to CGO'09 paper for issues concerning JIT compilation. Remove $\phi$-functions and perform renaming. Provide a more sophisticated pseudo-code (than for \ref{sec:classical_destruction}) for parallel copies sequentialisation (that can be performed before or after coalescing and $\phi$ removal.

