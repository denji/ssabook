       \chapter{Classical construction/update/destruction algorithm \Author{J. Singer, F. Rastello}}

\def\phiops{$\phi$-functions}
\def\phiop{$\phi$-function}
\section{Simple programming language model}

Simple description of an intraprocedural control flow graph
representation, with non-address-taken local variables.
Forward references to sections dealing with more SSA for complex
features like memory, concurrency, etc.
No constraints on control-flow structures in CFG
(unlike in syntax-directed SSA construction algorithms).


\section{Construction}
{\em We need the notion of CFG, $r$ the entry node. Need the notion of dominance, and (iterated)-dominance-frontier. Need the notion of reaching definition.}

In this chapter, each procedure is represented as a control flow graph (CFG). All node of the CFG is supposed to be reachable from the entry node $r$.

Transforming a code that has no \phiops\ to a code that has SSA form is called SSA construction. The classical construction algorithm for SSA form falls into two phases. 
First \phiops\ are inserted so that any use of a given variable is reached (without traversing another definition) by only one definition of this variable. In other words, this first phase of \phiops\ insertion splits live-ranges. The so obtained live-ranges have the property of having only one definition.
Then variables are renamed so that the code contains textually only one definition per variable, and such that each use refers to its corresponding unique reaching definition. In other words, this second phase assigns one variable per live-range. 

Historically, SSA came into different flavors. Building the so called minimal SSA (which is not completely minimal as we will see further) requires to insert \phiops\ at join nodes, i.e. nodes that can be reached by two different definitions of the same variable using two disjoint paths. 
% FAB: in fact Cytron et al. iterate on join sets. Weiss said that J+(S)=J(S). So removed the citation that was useless
Under the assumption that each variable has a {\em pseudo definition} at the root~$r$ of the CFG, this set of join nodes is exactly the iterated dominance frontier.
It ends up with a very simple algorithm that places \phiops\ variable per variable. 
For a given variable it places \phiops\ at the iterated dominance frontier $DF^{+}(S)$ of the set of definition points $S$ for this variable. 
Considering that the dominance frontier of each CFG node is precomputed, the \phiops\ can be inserted iteratively using a worklist of definition points, and flags (to avoid multiple insertions). 
This gives the following pseudo-code for \phiops\ insertion:
\begin{verbatim}
proc phi_insertion
// places phi functions at the iterated dominance frontier
  foreach global variable v
    let Worklist:=definitions(v)
    let Inserted:=emptyset
    foreach D in Worklist
      foreach I in DF(D) \ Inserted
        insert "v=phi(v,...,v)" @ beginning of node I
        Inserted += {I}
\end{verbatim}

Once \phiops\ are placed using this algorithm, the code may still contain several definitions per variables, but a use may refer to only one definition (that reach him). Moreover, with the following semantic for \phiops\ where its uses are considered to be on the corresponding predecessor blocks, then use-def chains are along the dominance tree. In other word the unique definition that reach a use dominates it. 

To obtain a code with a single assignment per variable, renaming have to be performed. But thanks to the dominance property this can be easily achieved using a DFS traversal of the dominance tree: 
during the traversal one just have to keep track for each variable, of its nearest dominating definition.
This gives the following pseudo-code for variable renaming:
\begin{verbatim}
proc renaming:
// rename variable def and use to have one definition per variable
  foreach v: Variable
    let v.reachingDef:=undef
  foreach B: basicBlock in DFS preorder of the dominance tree
    foreach inst: instruction from top to bottom of B
      foreach o: a use operand of inst
        let v the variable used in o
        v.update_reachingDef(o)
        replace v by v.reachingDef in o
      foreach o: a def operand of inst
        let v the variable used in o
        v.update_reachingDef(o)
        create new variable v'
        replace v by v' in o
        v'.reachingDef:=v.reachingDef
    foreach phi: phi instruction in a successor of B
      foreach o: a use operand of phi
        let v the variable used in o
        v.update_reachingDef(o)
        replace v by v.reachingDef in o
\end{verbatim}

\begin{verbatim}
proc v.update_reachingDef(o):
// simply go up along the reachingDef chain until dominance is fulfilled 
  let rd:=v.reachingDef
  while not (rd==undef || def(rd) dominates o)
    rd:=rd.reachingDef
  v.reachingDef:=rd
\end{verbatim}

As detailed and explained further, this simple algorithm ends up with a SSA form that is minimal (with the meaning given by Cytron et al.), that could be pruned (some dead \phiops\ are unnecessarily inserted), that is conventional (the transformation that renames back all $\phi$-related variables into a unique representative one, and removes all \phiops\ is a correct SSA-destruction algorithm), and has the dominance property (each variable use is dominated by its unique definition).

\section{Turning general SSA into conventional/pruned/dominance property}

Discussion regarding relationship between minimal and pruned SSA, etc.
$phi$-function removal can give pruned SSA but break
dominance properties, etc. Keep this section simple, and don't duplicate
material from Philip Brisk in previous chapter.
Ask Philip who may have some materials for the ``turning into SSA with dominance property'' part.


\section{Destruction \Author{F. Rastello}}
\label{sec:classical_destruction}

SSA form is a simple and efficient way of having sparse representation of some program information used by some code analysis and optimization. Once we are done with SSA based optimizations, or at least before code generation, one have to get rid of \phiops\ that are not machine instructions. This phase is called SSA destruction. 
Destructing a freshly constructed SSA code is easy. One just have to first rename back into a unique representative variable all $\phi$-related variables (called SSA-webs) that had initially the same name. Then each \phiop\ have all its operands identical and can be removed to coalesce the related live-ranges.
Finding SSA-webs can be performed efficiently using the classical union-find algorithm for merging sets:
\begin{verbatim}
proc find_webs
// find the ssa web of each variable
foreach variable v:
  ssaweb(v) := {v}
foreach instruction of the form "a=phi(a1,...,an)"
  foreach ai union(ssaweb(a),ssaweb(ai))
\end{verbatim}

An SSA form is said to be conventional if for each ssaweb, all its variables can be safely renamed into a unique representative one. In other words each SSA web should be interference free. While a freshly constructed SSA code is conventional, this may not be the case after optimizations such as copy folding have been performed.

The simplest (but not the more efficient) way for destructing non conventional SSA form is to split all (critical) edge, and then replace \phiops\ by parallel copies on predecessor blocks. By splitting an edge say ($s,d)$, we mean replacing it by an edge from $s$ to a freshly created basic block and by another edge from this new basic block to $d$.

{\em Provide a simple but not optimal pseudo code for parallel copy sequentialization. Provide the pseudo-code for $\phi$-function replacement.}

We should outline that this technique has several drawbacks: first because of specific architectural constraints, region boundaries, or exception handling code, the compiler might not allow to split a given edge; second, the obtained code contains many additional temporary-to-temporary copies. In theory, coping with these copies is the role of the coalescer during the register allocation phase. But only decoupled register allocators (such as the one detailed in Chapter~\ref{ chap:register-allocation}) with memory and time consuming coalescing heuristics can handle the removal of these copies efficiently. Coalescing can also, with less efforts, be handled prior to the register allocation phase. As opposed to a (so called conservative) coalescer during register allocation, this so called aggressive coalescing would not cope with the interference graph colorability. Still the insertion of copies itself might take a substantial amount of time and might not be suitable for dynamic compilation. The goal of Chapter~\ref{chap:alternative-destruction} is to cope both with non splittable edges and difficulties related with SSA destruction at machine code level, but also aggressive coalescing in the context of resource constrained compilation.


