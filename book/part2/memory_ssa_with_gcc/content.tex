\chapter{Memory SSA in GCC \Author{R. Guenther, D. Novillo}}
\chapterauthor{Guenther, Novillo}

\textbf{Total page count for chapter: 9}

\section{Overview}

\textbf{1 page}

Explain the different strategies used to represent aliasing and
memory in the IL.  Start with aggregates, which motivates the
representation of may-def/may-use information.  Continue with
pointer dereferences, the importance of having single-level
pointer dereferences in the IL.

\section{Representation of memory references}

\textbf{2 pages}

\begin{itemize}
\item	SSA suitable for scalars but not memory.  Explain
	approaches.

\item	Aliasing is undecidable in general.  Aliasing precision
	increases means more detailed tracking, which means more
	memory consumption and time consumption.

\item	Describe how GCC represents memory operations.
	Artificial symbols for aggregates and pointer
	dereferences.
\end{itemize}

\section{Memory SSA}

\textbf{6 pages}

\begin{itemize}
\item	Explain how all this gets out of hand quickly and the
	need for grouping heuristics.

\item	Describe grouping heuristics and what properties they
	should maintain.

\item	Describe the various grouping heuristics GCC used over
	time (inverting alias sets, dynamic partitioning, static
	partitioning, hybrid partitioning).

\item	Describe the problems with each one, leading to the
	current implementation.

\item	Describe aliasing oracle.
\end{itemize}


\section{Related Work}

\textbf{1 page}

SGI's approach is to use as many different symbols as possible
\cite{bib:chow.ea-96}, but they still link all the names for the
same symbol.  All the variables that are grouped under the same
symbol are always linked together.  Use-def chains over these
symbols are sparse, but every use is linked to every definition.

Steensgaard proposed assignment factoring
\cite{bib:steensgaard-95}.  It overlays factored use-def links
over the CFG.  No details on application to aggregate types or
global storage.  Store fragmentation (i.e., points-to analysis)
determines what statements conflict.  No details on forming the
SSA form nor keeping it up-to-date.

Cytron et.al. proposed iterating SSA formation using results of alias
analysis \cite{bib:cytron.ea-93}.  But the approach is too
expensive, it requires building the SSA form over and over.

Choi et.al. proposed location factoring \cite{bib:choi.ea-94}.
This is a precursor to Steensgaard's assignment factoring where
use-def links are inserted between statements that access the
same memory location.  It's the same idea, but less factored than
assignment factoring.

Fink et.al. proposed Heap SSA \cite{bib:fink.ea-00}.  Based on Array SSA
\cite{bib:knobe.ea-98}.
