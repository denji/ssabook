\begin{figure}
\begin{center}
\parbox{3.0in}{
\begin{programa}
\Ta $k$ := 2 \\
\Ta do $i$ := $\ldots$ \\
\Tb  $\ldots$ \\
\Tb  $a[i] := k * 5$ \\
\Tb  $\ldots := a[i]$ \\
\Ta enddo \\
\end{programa}
}
\end{center}
\caption{Example of Constant Propagation through Non-constant Index}
\label{fig:non-const-ex-source}
\end{figure}

In this section we address constant propagation
through {\it non-constant array subscripts}, as a generalization
of the algorithm
for constant subscripts described in section~\ref{sec:arraylattice}.
As an example, consider the program fragment in
figure~\ref{fig:non-const-ex-source}.  In the loop in
figure~\ref{fig:non-const-ex-source}, we see that the read access of
$a[i]$ will have a constant value ($k*5=10$), even though the
index/subscript value $i$ is not a constant.  We would like to extend
the framework from section~\ref{sec:arraylattice}  to be
able to recognize the read of $a[i]$ as constant in such programs.
There are two key extensions that need to be considered
for non-constant (symbolic) subscript values:
\begin{itemize}

\item  For constants, $C_1$ and $C_2$, $\DS$($C_1$, $C_2$) $\neq$ 
$\DD$($C_1$, $C_2$). However, for two symbols,  $S_1$ and $S_2$,
it is possible that both $\DS$($S_1$, $S_2$) and $\DD$($S_1$,
$S_2$) are FALSE, that is, we don't know if they are the same or different.

\item For constants, $C_1$ and $C_2$, the values for $\DS$($C_1$, $C_2$) and 
$\DD$($C_1$, $C_2$) can be computed by inspection. For symbolic
indices, however, some program analysis is necessary
to compute the $\DS$ and $\DD$ relations.
\end{itemize} 

We now discuss the compile-time
computation of $\DS$ and $\DD$ for symbolic
indices. 
Observe that,
given index values $I_1$ and $I_2$, only one of the following three cases
is possible:
\begin{center}
\parbox{3.0in}{
\begin{programa}
Case 1:   $\DS$($I_1$, $I_2$) = FALSE; $\DD$($I_1$, $I_2$) = FALSE \\
Case 2:   $\DS$($I_1$, $I_2$) = TRUE; $\DD$($I_1$, $I_2$) = FALSE \\
Case 3:   $\DS$($I_1$, $I_2$) = FALSE; $\DD$($I_1$, $I_2$) = TRUE \\
\end{programa}
}
\end{center}
The first case is the most conservative solution.  In the absence of
any other knowledge, it is always correct to state that
$\DS(I_1,I_2) = \mbox{\it false}$ and $\DD(I_1,I_2) = \mbox{\it
false}$.

The problem of determining
if two symbolic index values are the same is equivalent to the
classical problem of {\it global value numbering}. If two indices $i$
and $j$ have the same value number, then $\DS(i,j)$ must = {\it true}.
The problem of computing
$\DD$ is more complex. Note that $\DD$, unlike $\DS$, is not an
equivalence relation because $\DD$ is not transitive.
If $\DD(A, B)=\mbox{\it true}$
and $\DD(B, C)=\mbox{\it true}$, it does
not imply that $\DD(A, C)=\mbox{\it true}$.  
\REM{
Consider another type of
partitioning. Since for two indices, $I_1$ and $I_2$, if $\DS$($I_1$, $I_2$) then $I_1$ and $I_2$ are in the same value numbering
partition, we might attempt to partition the space of indices so that
if two indices are in different partitions they are $\DD$. However, given two indices known to be different (e.g., i and
i+1) in distinct partitions, there is no legal partition in which to
place j about which nothing is known.
}
However, we can leverage past work on array dependence analysis
to identify cases for which $\DD$ can be
evaluated to {\it true}.  For example, it is clear that $\DD(i,
i+1) = \mbox{\it true}$,
and that $\DD(i, 0) = \mbox{\it true}$ if $i$ is a loop index variable
that is known to be $\geq 1$.

Let us consider how the $\DS$ and $\DD$ relations for 
symbolic index values
are used by our constant
propagation algorithms.  Note that the specification of how $\DS$ and $\DD$
are used is a separate issue from the precision of
the $\DS$ and $\DD$ values.
We now describe how 
the lattice and the lattice operations presented in
section~\ref{sec:arraylattice} can be extended
to deal with non-constant subscripts.

First, consider the lattice itself. 
The 
$\top$ and $\bot$ lattice elements retain
the same meaning as in section~\ref{sec:arraylattice}
\viz\ $\Set(\top) = \{\;\}$ and 
$\Set(\bot) =  \U^A_{ind} \times \U^A_{elem}$.
Each element in the lattice is a list
of index-value pairs where the value is still required to
be constant but the index may be symbolic --- the index is
represented by its value number.

We now revisit the processing of an array element read of $A_1$[$k$] and
the processing of an array element write of $A_1$[$k$]. These
operations were presented in section~\ref{sec:arraylattice}
(figures~\ref{fig:aref} and \ref{fig:adef})
for constant indices. The versions for
non-constant indices appear in figure~\ref{fig:symb-aref} and
figure~\ref{fig:symb-adef}.
For the read operation in figure~\ref{fig:symb-aref}, if there exists a pair ($i_j$,$e_j$) such that
$\DS$($i_j$,$\Valnum(k)$) = {\it true} (\ie\ $i_j$ and $k$ have the
same value number), then the
result is $e_j$.  Otherwise, the result is $\top$ or $\bot$ as specified
in figure~\ref{fig:symb-aref}.
For the write operation in figure~\ref{fig:symb-adef}, if the value of the right-hand-side, $i$, is a constant, the result is the singleton list
$\langle(\Valnum(k),\L(i))\rangle$.  Otherwise, the result is $\top$ or $\bot$ as specified
in figure~\ref{fig:symb-adef}.

\begin{figure}%[h]
\begin{center}
\begin{tabular}{|l||c|c|c|}
\hline
$\L(A_1[k])$ & $\L(k) = \top$ & $\L(k) = \Valnum(k)$ & $\L(k) = \bot$ \\
\hline \hline
$\L(A_1) = \top$ & $\top$ & $\top$ & $\bot$ \\
\hline
$\L(A_1) = \langle (i_1,e_1), \ldots \rangle$ & $\top$ & $e_j$, 
if $\exists$
$(i_j,e_j) \in \L(A_1)$ with &\\
& & $\DS(i_j, \Valnum(k)) = \mbox{\it true}$ & $\bot$\\
& & $\bot$, otherwise & \\
\hline
$\L(A_1) = \bot$ & $\bot$ & $\bot$ & $\bot$ \\
\hline
\end{tabular}
\end{center}
\caption{Lattice computation for \protect{$\L(A_1[k]) = \L_{[\:]}(\L(A_1), \L(k))$},
where $A_1[k]$ is an 
array element read operator. If $\L(k)=\Valnum(k)$, the lattice value of index $k$ is a value number that represents a constant or a symbolic value.}
\label{fig:symb-aref}
\end{figure}

\begin{figure}%[h]
\begin{center}
\begin{tabular}{|l||c|c|c|}
\hline
$\L(A_1)$ & $\L(i) = \top$ & $\L(i) = Constant$ & $\L(i) = \bot$ \\
\hline \hline
$\L(k) = \top$ & $\top$ & $\top$ & $\bot$ \\
\hline
$\L(k) = \Valnum(k)$ & $\top$ & $\langle(\Valnum(k),\L(i))\rangle$ & $\bot$ \\
\hline
$\L(k) = \bot$ & $\bot$ & $\bot$ & $\bot$ \\
\hline
\end{tabular}
\end{center}
\caption{Lattice computation for \protect{$\L(A_1) = \L_{d[\:]}(\L(k), \L(i))$},
where \protect{$A_1[k] := i$} is an 
array element write operator. If \protect{$\L(k)=\Valnum(k)$}, the lattice value of index \protect{$k$} is a value number that represents a constant or a symbolic value.}
\label{fig:symb-adef}
\end{figure}





Let us now consider
the propagation of lattice values through $d\phi$
operators.
The only extension required  relative to figure~\ref{fig:dphi} is that
the $\DD$ relation used in performing
the \Update\ operation should be able to determine when
$\DD(i',i_j)=\mbox{\it true}$ if $i'$ and $i_j$ are symbolic
value numbers rather
than constants.  (If no symbolic information is available for $i'$
and $i_j$, then it is always safe
to return $\DD(i',i_j)=\mbox{\it false}$.)

% Note that the $\Update$ operation (section~\ref{sec:arraylattice},
% page~\pageref{def:update}) always returns a non-empty list,
% even if it uses the most conservative $\DD = false$ approach.
% Further, the list will always 
% contain the pair $(i',e')$, so long as the heuristic
% for dropping a pair to obey the $\leq Z$ size limit chooses
% a pair other than $(i',e')$ to drop.
% This is sufficient to handle
% the propagation of the constant through the symbolic index in the
% example in figure~\ref{fig:non-const-ex-source}.
% More precise computations of the $\DD$ relation will lead to 
% more precise (longer) lists, and hence lead to discovery of
% more
% constants  for more complicated programs with symbolic subscripts.








