
\begin{figure}
{\small
\begin{tabular}[t]{ll}
\begin{minipage}[t]{2.0in}
\setlength{\baselineskip}{10pt}
Original program: \\
%-----------------
\begin{verbatim}
  p  := new Type1
  q  := new Type1
  . . .
 p.x := ...
 q.x := ...
...  := p.x

\end{verbatim} 
\bigskip
After redundant load elimination: \\
%---------------
\begin{verbatim}
  p  := new Type1
  q  := new Type1
  . . .
  T1 := ...
 p.x := T1
 q.x := ...
...  := T1
\end{verbatim} 
\begin{center}
(a)
\end{center}
\end{minipage}
&
\begin{minipage}[t]{2.0in}
\setlength{\baselineskip}{10pt}
Original program: \\
\begin{verbatim}
  p  := new Type1
  q  := new Type1
  . . .
...  := p.x 
 q.x := ...
...  := p.x

\end{verbatim} 
\bigskip
After redundant load elimination: \\
%---------------
\begin{verbatim}
  p  := new Type1
  q  := new Type1
  . . .
  T2 := p.x
...  := T2
 q.x := ...
...  := T2
\end{verbatim} 
\begin{center}
(b)
\end{center}
\end{minipage}
% &
% \begin{minipage}[t]{2.0in}
% \setlength{\baselineskip}{10pt}
% Original program: \\
% \begin{verbatim}
%   p  := new Type1
%   q  := new Type1
%   r  := p
%   . . .
%  p.x := ...
%  q.x := ...
%  r.x := ...
% \end{verbatim} 
% \bigskip
% After dead store elimination: \\
% %---------------
% \begin{verbatim}
%   p  := new Type1
%   q  := new Type1
%   r  := p
%   . . .
%  q.x := ...
%  r.x := ...

% \end{verbatim} 
% \begin{center}
% (c)
% \end{center}
% \end{minipage}
\end{tabular}
}
\caption{Examples of scalar replacement}
\label{fig:ex2}
\end{figure}

% In this section, we introduce two new analyses based on Extended
% Array SSA form.  These two analyses form the backbone of 
% {\em scalar replacement} transformations, which replace accesses
% to memory by uses of scalar temporaries.  First, we present an
% analysis to identify fully redundant loads.  Then, we present an
% analysis to identify dead stores.

Figure~\ref{fig:ex2} illustrates two different cases of scalar
replacement for object fields. 
For the original program in figure~\ref{fig:ex2}(a),
introducing a scalar temporary {\tt T1} for the store (def) of {\tt
p.x} can enable the load (use) of {\tt p.x} to be eliminated \ie\ to
be replaced by a use of {\tt T1}.  Figure~\ref{fig:ex2}(b)
contains an example in which a scalar temporary ({\tt T2}) is
introduced for the first load of {\tt p.x}, thus enabling the second
load of {\tt p.x} to be eliminated \ie\ replaced by {\tt T2}.  % Finally,
% figure~\ref{fig:ex2}(c) contains an example in which the first store of {\tt
% p.x} can be eliminated because it is known to be dead (redundant); no
% scalar temporary needs to be introduced in this case.
