% Next, section~\ref{sec:scc} presents an extension to the Sparse Conditional
% Constant propagation (SCC) algorithm from \cite{WeZa91} that enables
% constant propagation through array elements in conjunction with 
% unreachable code elimination. 

In this chapter,
we introduced an Array SSA form 
that captures element-level data-flow information for array variables,
illustrated how it can be used to extend program analyses for
scalars to array variables using constant propagation 
as an exemplar, and illustrated how it can be used to extend
optimizations for scalars to array variables using load elimination in
heap objects as
an exemplar.
In addition to reading the other chapters in this book for related
topics, the interested reader can consult \cite{KnSa98} for 
details on full Array SSA form, \cite{KnSa98b} for details on constant
propagation using Array SSA form, \cite{FiKS00} for
details on load and store elimination for pointer-based objects using
Array SSA form, \cite{SaFi01} for efficient dependence analysis of
pointer-based array objects, and \cite{BaSa09} for extensions of this load
elimination algorithm to parallel programs.


There are many possible directions for future research based on this
work.
The \ds\ and \dd\ analyses outlined in this chapter are sound, but
conservative.  In restricted cases, they can be made more precise using array subscript
analysis from polyhedral compilation frameworks.  Achieving a robust
integration of Array SSA and polyhedral approaches is an interesting goal for future
research.  Past work on Fuzzy Array Data-flow Analysis
(FADA)~\cite{BCF97} may provide a useful foundation for exploring such
an integration.
Another interesting direction is to extend the value numbering and 
definitely-different analyses mentioned in section~\ref{sec:non-const}
so that they can be combined with constant propagation
rather than performed as a pre-pass.
An ultimate goal is
to combine conditional constant, type propagation, value numbering, partial redundancy elimination,
and scalar replacement analyses within a single framework that can be
used to analyze scalar variables, array variables, and pointer objects
with a unified approach.
