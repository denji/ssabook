% TODO:
% 1) Replace earlier Array SSA example by the same one used for
% constant propagation

% The goal of Array SSA form is to provide the same benefits for arrays
% and related data structures
% that traditional SSA form provides for scalars.
% Section~\ref{sec:full} summarizes  {\it full Array SSA form}, which provides exact use-def information 
% at run-time for each dynamic access of an array element.
% Section~\ref{sec:partial} introduces {\it partial Array SSA form} as a
% static approximation of full Array SSA form.
% Throughout this chapter,
% we assume that all array operations in the input program are
% expressed as reads and writes of individual array elements.  The
% extension to more complex data structures such as arrays of
% structures and nested arrays 
% has been
% omitted to simplify the presentation of this chapter.

%\subsection{Full Array SSA Form}\label{sec:full}

% The full Array SSA form uses $\Phi$ operators instead of $\phi$
% functions used by traditional SSA form. The semantics of the $\Phi$
% operator can be defined as a pure function.  
% This is one respect in which Array SSA form
% has
% advantages over traditional SSA form even for scalar variables.
% {\it @ variables} (pronounced ``at variables'')
% are used to obtain a pure function semantics for the
% $\Phi$
% operator. Each
% $\phi$ function in traditional SSA form such as $\phi(S_1,S_2)$ is
% rewritten as $\Phi(S_1,@S_1,S_2,@S_2)$.  

To introduce full Array SSA form with runtime evaluation of $\Phi$
functions,
we use the concept of an  {\it iteration vector} to differentiate among multiple dynamic
instances of a static definition, $S_k$, that occur in the same
dynamic instance of $S_k$'s enclosing procedure, $f()$.
% The {\it iteration vector} of a 
% static 
% definition $S_k$ identifies a single iteration in the iteration space of the
% set of loops
% that enclose the definition. 
Let $n$ be the number of loops that enclose $S_k$ in procedure $f()$.
These loops could be for-loops, while-loops, or even loops constructed
out of goto statements.
For convenience, we treat the outermost
region of acyclic control flow in a procedure as a dummy outermost loop
with a single iteration, thereby ensuring that $n \geq 1$.

A single point in the
iteration space is specified by the iteration vector
$\vec{i} = (i_1, \ldots, i_n)$, which is
an 
$n$-tuple of iteration numbers,
<one for each enclosing loop. 
For convenience, this definition of iteration vectors assumes that  
all loops are single-entry, or equivalently, that the control-flow graph is {\it reducible}.
(This assumption is not necessary
for partial Array SSA form.)
For single-entry loops, we know that each def executes at most
once in a given iteration of its surrounding loops, hence the iteration vector
serves the purpose of a ``timestamp''.
The key extensions in Array SSA form relative to standard SSA form are as
follows.



\begin{enumerate}
\item {\bf Renamed array variables:}
All array variables are renamed so as to 
satisfy the static single assignment property.  Analogous to standard SSA
form, control $\Phi$ operators are introduced to generate new names
for merging two or more prior definitions at control-flow join points, and to ensure that each use
refers to precisely one definition.

\item {\bf Array-valued @ variables:}
For each static definition
$A_j$, we introduce an {\em @ variable} (pronounced ``at variable'')
$@A_j$ that identifies
the most recent {\em iteration vector} {\it
at} which definition $A_j$ was executed.
We assume that all @ variables are initialized to the empty
vector, $ (\;)$, at the start of program execution.  
% For each
% real (non-$\Phi$) definition of a renamed scalar, $S_k$, we assume that a statement of the
% form $@S_k := \vec{i}$ is inserted immediately after definition
% $S_k$, where $\vec{i}$
% is the current iteration vector for all loops that surround
% $S_k$. 
% All @ variables are initialized
% to the empty vector because the empty vector is the identity element
% for a lexicographic $\max$ operation \ie\ $\max((\;),\vec{i}) =
% \vec{i}$, for any @ variable value $\vec{i}$.
% Each  array variable, $A_j$, in Array SSA form has an associated 
% @ variable, $@A_j$, such that $@A_j$ has
% the same shape (rank and dimension sizes) as array variable $A_j$.
Each update of a single array element, $A_j[k] := \ldots$, 
is followed by the statement, $@A_j[k] := \vec{i}$
where $\vec{i}$ is the iteration vector for the loops surrounding
the definition of $A_j$.
% Thus, an array-valued @ variable, $@A_j$,  
% can record a separate iteration vector for
% each element that is assigned by definition $A_j$.


\item {\bf Definition $\Phi$'s:}
\label{def:phi}
A  {\it definition}-$\Phi$ operator is 
introduced in Array SSA form to deal with preserving (``non-killing'') definitions
of arrays.  Consider $A_0$ and $A_1$, two renamed 
arrays that originated from the same array variable in the source program
such that $A_1[k] := \ldots$
is an update of a single array element
and $A_0$ is the prevailing definition at the program point just
prior to the definition of $A_1$.
A definition $\Phi$, $A_2 := d\Phi(A_1, @A_1, A_0, @A_0)$,
is inserted immediately after the definitions for $A_1$ and $@A_1$.
% (We use the notation $d\Phi$ when we want to 
% distinguish a definition $\Phi$ operator from a control $\Phi$ operator.)
Since definition $A_1$ only updates one element of $A_0$, $A_2$ represents
an element-level merge of arrays $A_1$ and $A_0$.
Definition $\Phi$'s did not need to be
inserted in standard SSA form because a scalar definition completely kills the old value of
the variable.  


\item {\bf Array-valued  $\Phi$ operators:}
\label{array:phi}
Another consequence of renaming arrays is that
a $\Phi$ operator
for array variables must also return an
array value.  Consider a (control or definition) $\Phi$ operator of
the form, $A_2 := \Phi(A_1, @A_1, A_0, @A_0)$. Its semantics can be specified precisely
by the following conditional expression
for each element, $A_2[j]$, in the result array $A_2$:
\begin{eqnarray}
A_2[j] & = &
  \begin{array}{llll}
\mbox{\bf if} & @A_1[j] \succeq @A_0[j] & \textbf{then} & A_1[j] \\
\mbox{\bf else} & A_0[j] \\
\mbox{\bf end if} \label{eqn:cond-expr}
  \end{array}
\end{eqnarray}
The key extension over the scalar case is that the conditional expression
specifies an element-level merge of arrays $A_1$ and $A_0$.
\end{enumerate}





Figures~\ref{fig:ssa-acyclic-array} and \ref{fig:full-form}
show an example program with an
array variable, and the conversion of the program to full Array SSA form as
defined above.

\begin{figure}%[p]
\begin{center}
\parbox{3.0in}{
\begin{programa}
%\mbox{n1:}
\Tb $A[*] := \mbox{\rm initial value of $A$}$\\
\Tb$i := 1$ \\
\Tb $C := i\ <\ 2 $\\
\Tb if $C$ then \\
%\mbox{n2:}
\Tc $k := 2\ *\ i$ \\
\Tc $A[k] := i$\\
\Tc print $A[k]$\\
\Tb endif \\
%\mbox{n3:}
\Tb print $A[2]$
\end{programa}
}\\
\end{center}
\caption{Example program with array variables}
\label{fig:ssa-acyclic-array}
\end{figure}


\begin{figure}%[p]
\begin{center}
\parbox{3.0in}{
\begin{programa}
%\mbox{n1:}
%\Tb $@i := (\;)$ ; $@C := (\;)$ ; $@k := (\;)$ ; \\
\Tb $@A_0[*] := (\;)$ ; $@A_1[*] := (\;)$\\
\\
\Tb $A_0[*] := \mbox{\rm initial value of $A$}$\\
\Tb $@A_0[*] := (1)$\\
\Tb $i := 1$ \\
%\Tb $@i := (1)$ \\
\Tb $C := i\ <\ n $ \\
%\Tb $@C := (1)$ \\
\Tb if $C$ then \\
%\mbox{n2:}
\Tc $k :=  2\ *\ i$ \\
%\Tc $@k := (1)$ \\
\Tc $A_1[k] := i$\\
\Tc $@A_1[k] := (1)$\\
\Tc $A_2 := d\Phi(A_1, @A_1, A_0, @A_0)$\\
\Tc $@A_2 := \max(@A_1, @A_0)$\\
\Tc print $A_2[k]$\\
\Tb endif \\
%\mbox{n4:} 
\Tb $A_3 := \Phi(A_2, @A_2, A_0, @A_0)$\\
\Tb $@A_3 := \max(@A_2, @A_0)$\\fig:
\Tb print $A_3[2]$ 
\end{programa}
}
\end{center}
\caption{Conversion of program in figure \protect{\ref{fig:ssa-acyclic-array}} to Full Array SSA Form}
\label{fig:full-form}
\end{figure}

We now introduce a {\it partial Array SSA form} for static analysis,
that serves as an approximation of full Array SSA form.
Consider a (control or definition) $\Phi$ statement, $A_2 := \Phi(A_1, @A_1, A_0, @A_0)$.
A static analysis will need to 
approximate
the computation of this $\Phi$ operator by 
some data-flow transfer function, $\L_{\Phi}$.
The inputs and output of $\L_{\Phi}$ will be
{\it lattice elements} for scalar/array variables that
are compile-time approximations of their run-time values.
We use the notation $\L(V)$ to denote the lattice element for 
a scalar or array
variable $V$.
Therefore, the 
statement, $A_2 := \Phi(A_1, @A_1, A_0, @A_0)$, will in general
be modeled by the data-flow equation,
$\L(A_2) = \L_{\Phi}(\L(A_1), \L(@A_1), \L(A_0), \L(@A_0))$.

While the  {\em runtime} semantics of 
$\Phi$ functions for array variables critically depends on @ variables (Equation~\ref{eqn:cond-expr}),
many {\em compile-time analyses} do not need the full generality of @ variables.  
For analyses that do not distinguish among iteration instances,
it is sufficient to model
$A_2 := \Phi(A_1, @A_1, A_0, @A_0)$ by
a data-flow equation, $\L(A_2) = \L_{\phi}(\L(A_1), \L(A_0))$,
that does not use lattice variables $\L(@A_1)$ and $\L(@A_0)$.
For such cases, a {\it partial}
Array SSA form can be obtained by dropping 
dropping @ variables, and using the
$\phi$ operator, $A_2 := \phi(A_1, A_0)$ instead of
$A_2 := \Phi(A_1, @A_1, A_0, @A_0)$.  
A consequence of dropping @ variables is that partial Array
SSA form does not need to deal with iteration
vectors, and therefore does not require the control-flow graph to be {\it reducible} as in full Array SSA form.
For scalar variables, the resulting $\phi$ operator obtained by
dropping @ variables exactly coincides with standard SSA form.
% The use of $\phi$ operators without @ variables brings
% partial Array SSA form closer to traditional SSA form.
% The key difference is that partial Array SSA form still contains 
% renamed arrays and definition $\phi$ operators for updates to array elements.



% Our first observation is that there is no extra information
% provided at compile-time by the @ variables
% for any static analysis that does not distinguish between reachable code
% and unreachable code.  In such cases,
% it is sufficient to model
% $A_2 := \Phi(A_1, @A_1, A_0, @A_0)$ by
% a data-flow equation of the form $\L(A_2) = \L_{\phi}(\L(A_1), \L(A_0))$
% that does not use lattice variables $\L(@A_1)$ and $\L(@A_0)$.
% For array variables, the only useful information
% provided by an @ variable, $@A_1$ (say), at compile-time is an indication
% of which elements were updated by the assignment to array $A_1$.  
% %The actual ``timestamp'' (iteration vector) for the assignment is not
% %relevant if all paths are considered to be reachable.
% However, as we will see in section~\ref{sec:arraylattice}, this information
% is also included in the lattice value $\L(A_1)$ for array $A_1$. 

% Our second observation is that
% a static analysis that needs to distinguish between unreachable
% code and reachable code can do so efficiently
% by introducing {\it executable flags} for nodes and edges in the
% CFG (Control-flow Graph) as i.
% If executable flags are computed in the data-flow analysis, then
% the @ variables again do not provide any useful extra information.
% In fact, if we consider a control $\Phi$ statement
% $A_2 := \Phi(A_1, @A_1, A_0, @A_0)$, with array values $A_1$ and $A_0$
% carried by incoming CFG edges $e1$ and $e0$ respectively, then
% the corresponding data-flow equation (in the presence of unreachable code
% elimination) will be 
% $\L(A_2) = \L_{\Phi}(\L(A_1), X_{e1}, \L(A_0), X_{e0})$ where 
% $X_{e1}$ and $X_{e0}$ are the executable flags for edges $e1$ and $e0$.
% (Full details can be found i.)

% Since @ variables need not be modeled for the
% compile-time analyses discussed in this chapter,
% we drop them and use the
% $\phi$ operator, $A_2 := \phi(A_1, A_0)$ instead of
% $A_2 := \Phi(A_1, @A_1, A_0, @A_0)$ in {\it partial}
% Array SSA form.  
% A consequence of dropping @ variables is that partial Array
% SSA form does not need to deal with iteration
% vectors, and therefore does not require the control-flow
% graph to be {\it reducible} as in full Array SSA form.
% The use of $\phi$ operators without @ variables brings
% partial Array SSA form closer to traditional SSA form.
% The key difference is that partial Array SSA form still contains 
% renamed arrays and definition $\phi$ operators for updates to array elements.

