We present experimental results using the \jp\ optimizing compiler~\cite{grande99}.
\REM{
\jp\ uses a {\em compile-only} methodology for running Java code; the
system translates all Java bytecodes to native machine instructions, with
no interpreter.  A key feature of \jp\ is that the entire virtual
machine, including the thread scheduler, memory management systems,
class loader, and compilers, is implemented in Java.  
}
The performance results in this section were obtained on an
IBM F50 Model 7025 system
with four 166MHz PPC604e processors running AIX v4.3.
However, the results presented in this section were
obtained for single-threaded programs.
The system has 1GB of  main memory. 
Each processor has split 32KB first-level instruction and data
caches and a 256KB second-level cache. 

The \jp\ system is continually under development; the results in this
section use the \jp\ system as of November~8, 1999.  For these experiments,
the \jp\ optimizing compiler performed a basic set of 
standard optimizations
including copy propagation, type propagation, null check elimination, 
constant folding, devirtualization, local common subexpression elimination,
load/store elimination, dead code
elimination, and linear scan register allocation.  Previous work~\cite{grande99} has
demonstrated that \jp\ performance with these optimizations is roughly
equivalent to that of the industry-leading IBM product JVM and JIT.
The runs use \jp`s non-generational copying
garbage collector with 300MB of heap (which is shared by the application
and by all components of the \jp\ JVM).


Our preliminary implementation has several limitations.  
Our current implementation does 
not eliminate the null-pointer and array-bounds checks for redundant loads. 
We do not use any interprocedural
summary information, as the \joc\ assumes on ``open-world'' due to
dynamic class loading.  
We do not perform
any loop-invariant code motion or partial redundancy elimination 
to help eliminate redundant loads in loops.  Most importantly,
the \joc\ still lacks many classical scalar optimizations, which are
especially important to eliminate the register copies and reduce register
pressure introduced by scalar replacement.  For these reasons,
these experimental results should be considered a lower bound on
the potential gains due to scalar replacement, and we expect the
results to improve as \jp\ matures.

\REM{
Our implementation does not enforce
the coherence property of the Java Memory model, which does not
affect correctness of our generated code for single-threaded benchmarks.
}

Note that since the entire \jp\ JVM is implemented in Java, the optimizing
compiler compiles not only application code and library code, but also
VM code.  The results in this section thus reflect the performance
of the entire body of Java code which runs an application, which includes
VM code and libraries. Furthermore, the compiler inlines code across
these boundaries.
%; for example, the compiler inlines much of the VM 
%implementation directly into a user code method body.

For our experiments, we use the six serial codes from the 
SPECjvm98 suite~\cite{specJVM98}, and the Symantec suite of 
compiler microbenchmarks.  For the SPEC codes, we use the
medium-size (-s10) inputs.  Note that this methodology does {\em not}
follow the official SPEC run rules, and these results are not
comparable with any official SPEC results.  
The focus of our measurements was on obtaining dynamic counts
of memory operations.
When we report timing
information, we report the best wall-clock time from three runs.

\ssainput{part2/array_ssa/characterTable}

We instrumented the compiled code to count each of the six types of Java memory
operations eligible for optimization by our scalar replacement
algorithms: {\tt getfield, putfield, getstatic, putstatic, aload} and 
{\tt astore}.  Table~\ref{table:character} shows the dynamic count of
each operation during a sample run of each program.

\ssainput{part2/array_ssa/countTable}

Table~\ref{table:counts} shows the percentage of each type of memory
operation eliminated by our scalar replacement algorithms.  
The table shows that
overall, the algorithms eliminate between 0.7\% and 23.5\% of the
loads and stores.  The table shows that redundant load elimination
eliminates many more operations than dead store elimination.  On two
codes, (mpegaudio and symantec), elimination of loads from arrays play
a significant role.  On the others, the algorithm eliminates mostly
getfields.  Interestingly, the algorithms are ineffective at eliminating
references to static variables; however, table~\ref{table:character} shows
that these references are relatively infrequent.

\REM{
Note that surprisingly, some categories in table~\ref{table:counts} are
slightly negative; ie., the effect of scalar replacement actually
increased the dynamic count of some memory operations.  Scalar replacement
did not actually introduce any loads or stores.  This phenomenon can
be explained by recalling that the entire Jalapeno JVM is written
in Java, including some non-deterministic systems such as the thread
scheduler and garbage collector.  The increased counts in the table
represent small random variations from these systems, 
and are not statistically significant.
}

\ssainput{part2/array_ssa/speedupTable}

Table~\ref{table:speedup} shows the improvement in running time due
to the scalar replacement algorithm.  The results show that the
scalar replacement transformation has a large impact on execution-time,
increasing bottom-line performance by speedups of at least
$1.05\times$ on each of the 
four codes where we could eliminate over 10\% of the memory accesses.
{\tt Mpegaudio} shows the greatest improvement with a 
speedup\footnote{Note that a speedup of $1.23\times$ is equivalent to
a reduction in execution time of 19\%.} of $1.23\times$.

We conclude this section with a brief discussion of the impact of
scalar replacement on register allocation.  It has been observed in
the past that scalar replacement can increase register
pressure~\cite{CaKe94}.  For example, the scalar replacement
transformations shown in Figure~\ref{fig:ex2}(a) and
Figure~\ref{fig:ex2}(b) eliminate load instructions at the expense of
introducing temporaries with long live ranges.  This extra register
pressure ought not to pose a problem to a register allocator with
smart heuristics for live range splitting.  However, the register
allocation algorithm currently implemented in the \joc\ does not
perform any live range splitting.  In our initial experiments, we
observed that scalar replacement caused an excessive number of spills
for one method in {\tt db} and one method in {\tt mpegaudio}.  For the
results reported above, we hardwired the compiler to disable scalar
replacement for these two methods.  In future work, we will either
implement live range splitting so as to better tolerate the extra
register pressure created by scalar replacement, or automatically
disable scalar replacement in pathological cases.
