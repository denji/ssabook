This section shows how we extend the Sparse
Constant propagation (SC) algorithm from \cite{WeZa91} so as to enable
constant propagation through array elements.
For simplicity, this algorithm is restricted
to cases in which both the subscript and the value of an
array definition are constant.  A generalization to non-constant
subscripts is presented next in section~\ref{sec:non-const}.


\begin{figure}
\begin{center}
\begin{programa}
/* INITIALIZATION */ \\
\\
\Ta {\bf for each} lattice variable $\L(v)$ {\bf do}\\
\Tb {\bf if} it is not possible that $\L(v)$ can be recognized as a constant\\
\Td at compile-time (\eg\ $v$ is a return value from an unknown call)\\
\Tb {\bf then}\\
\Tc $\L(v)\  \leftarrow\ \bot $\\
\Tb {\bf else}\\
\Tc $\L(v)\  \leftarrow\ \top $\\
\Tb {\bf end if}\\
\Ta {\bf end for}\\
\\
\Ta Intialize {\it worklist} $\leftarrow$ an empty list\\
\Ta {\bf for each} equation $E$ {\bf do}\\
\Tb {\bf if} the RHS of equation $E$ has at least one term that is $\not= \; \top$ \\
\Td  (\ie\ at least one term that is $\bot$ or contains a constant)\\
\Tb {\bf then}\\
\Tc  Insert equation $E$ into the {\it worklist}\\
\Tb{\bf end if}\\
\Ta {\bf end for}\\
 \\
/* FIXPOINT ITERATION */ \\
\\
\Ta {\bf while} {\it worklist} is not empty {\bf do}\\
\Tb $E \leftarrow$ remove any equation from {\it worklist}\\
\Tb /* Heuristic: remove an equation that depends on the smallest number\\
\Tb ~~~of other equations in {\it worklist} */ \\
\Tb Recompute $\L(v)$, the LHS of equation $E$, based on the values of $E$'s RHS terms\\
\Tb {\bf if} the LHS of equation $E$ has changed {\bf then}\\
\Tc {\bf for each} equation $E'$ that uses $\L(v)$ in its RHS {\bf do}\\
\Td Insert equation $E'$ into {\it worklist}\\
\Tc {\bf end for}\\
\Tb {\bf end if}\\
\Ta {\bf end while} \\
\\
/* TERMINATION */\\
\\
\Ta For each Array SSA variable $v$ such that $\L(v)$ contains a constant, transform the \\ 
\Ta program to replace each use of $v$ by a constant, if profitable to do so
\end{programa}
\end{center}
\caption{Algorithm for Sparse Constant propagation (SC) for scalar and array variables}
\label{fig:sc-alg}
\end{figure}

Figure~\ref{fig:sc-alg} contains an outline of the sparse constant
propagation algorithm for scalar and array variables.  It is similar in
structure to the sparse constant propagation algorithm for scalars
presented in \cite{WeZa91}.  A major difference is that the data flow
equations now include the lattice values for array variables, as
described in section~\ref{sec:arraylattice}.
In addition, the algorithm in figure~\ref{fig:sc-alg} uses a worklist
of data flow equations (rather than a worklist of SSA edges as in \cite{WeZa91}), thus enabling future integration with other analysis algorithms that are
based on data flow equations.

Let us consider how the algorithm in figure~\ref{fig:sc-alg}
will work on the data flow equations in figure~\ref{fig:sc-ex-df}.
Recall that the Array SSA form for this example program 
(see figure~\ref{fig:sc-ex-ssa})
includes the propagation of array $Y_2$ into two control
flow successors, and the propagation into array $D_5$
from two control flow
predecessors. 


\begin{figure}
\begin{center}
\begin{tabular}{r r c l}
\mbox{S1:} & $\L(Y_1)$ & = & $\langle \; (3,99)\; \rangle$ \\
\mbox{S2:} & $\L(Y_2)$ & = & $\langle \; (3,99)\; \rangle$ \\
\mbox{S3:} & $\L(D_1)$ & = & $\langle \; (1,198)\; \rangle$ \\
\mbox{S4:} & $\L(D_2)$ & = & $\langle \; (1,198)\; \rangle$ \\
\mbox{S5:} & $\L(D_3)$ & = & $\bot$\\
\mbox{S6:} & $\L(D_4)$ & = & $\bot$\\
\mbox{S7:} & $\L(D_5)$ & = & $\bot$\\
\mbox{S8:} & $\L(Z)$ & =  & $\bot$
\end{tabular}
\end{center}
\caption{Solution to data flow equations from figure \protect{\ref{fig:sc-ex-df}}, assuming $I$ is unknown}
\label{fig:sc-ex-sol1}
\end{figure}

\begin{figure}
\begin{center}
\begin{tabular}{r r c l}
\mbox{S1:} & $\L(Y_1)$ & = & $\langle \; (3,99)\; \rangle$ \\
\mbox{S2:} & $\L(Y_2)$ & = & $\langle \; (3,99)\; \rangle$ \\
\mbox{S3:} & $\L(D_1)$ & = & $\langle \; (1,198)\; \rangle$ \\
\mbox{S4:} & $\L(D_2)$ & = & $\langle \; (1,198)\; \rangle$ \\
\mbox{S5:} & $\L(D_3)$ & = & $\langle \; (1,198)\; \rangle$ \\
\mbox{S6:} & $\L(D_4)$ & = & $\langle \; (1,198)\; \rangle$ \\
\mbox{S7:} & $\L(D_5)$ & = & $\langle \; (1,198)\; \rangle$ \\
\mbox{S8:} & $\L(Z)$ & =  & 198
\end{tabular}
\end{center}
\caption{Solution to data flow equations from figure \protect{\ref{fig:sc-ex-df}}, assuming $I$ is known to be = 3}
\label{fig:sc-ex-sol2}
\end{figure}

The initialization step first initializes $\L(Y_0)$
and $\L(D_0)$ to $\bot$, since they are both considered to be unknown
variables in this example program.  Next, {\it worklist} is intialized
to contain equations S1, S2, S3, S4, S5, S6, S8, since they all
contain some terms that are different from $\top$ \ie\ terms that
contain a constant or equal $\bot$.

For the fixpoint iteration step, the heuristic in
figure~\ref{fig:sc-alg} causes the equations in {\it worklist} to be
processed in their textual order \ie\ S1, S2, S3, $\ldots$.  
(This heuristic
is similar to the heuristic of processing basic blocks in 
a topological order of their positions in the control flow graph.)
The
iteration terminates when no LHS lattice variable changes its value,
thus causing {\it worklist} to become empty.  The final values of
the lattice variables obtained
after the fixpoint iteration step has completed are shown in
figure~\ref{fig:sc-ex-sol1}.


The above lattice values were obtained assuming $\L(I) = \bot$.  If, instead,
variable $I$ is known to equal 3 \ie\ $\L(I) = 3$, then 
the lattice variables that would be obtained
after the fixpoint iteration step has completed are shown in
figure~\ref{fig:sc-ex-sol2}.

In either case ($\L(I) = \bot$ or $\L(I) = 3$), the resulting constants
revealed by the algorithm can be used in whatever analyses or
transformations the compiler considers to 
be profitable to perform.  This is the job
of the termination
step in figure~\ref{fig:sc-alg}.

\REM{
Equations S1 and S2 are evaluated first because they are
associated with the entry block. Element 3 of $Y_2$ is known to have the
value 99 at this stage. As a result of the modification of $Y_2$ both S3
and S5 are inserted into the worklist (they reference the lattice
value of $Y_2$). S3 uses the propagated constant 99 to compute and
propagate the constant 198 to element 1 of $D_1$ and then, after
evaluation of S4, to element 1 of $D_2$.  Any subsequent references to D
in the {\tt then} block of the source become references $D_2$ in the
Array SSA form and are known to have a constant value at element 1.
Depending on the order of computations via the worklist, we may then
compute either $D_3$ and then $D_4$ or, because $D_2$ has been modified, we may
compute $D_5$. If we compute $D_5$ at this point, it appears to have a
constant value. Subsequent evaluations $D_3$ and $D_4$ cause $D_5$ to be
reevaluated and lowered from constant value to $\bot$ because the
value along one path is not constant.







Notice that in this case, the reevaluation of $D_5$ could have been
avoided by choosing an optimal ordering of processing.  Processing of
programs with cyclic control flow is no more complex but may involve
recomputation that can not be removed by optimal reordering. In
particular, the loop entry is a control flow merge point since control
may enter from the top or come from the loop body. It will contain a
$\phi$ which combines the value entering from the top with that
returning after the loop. The lattice values for such a node may
require multiple evaluations.

Also notice that in this example, if $I$ in S5 is known to have the
value 3, it will be recoded as a constant element. Upon evaluation of S7, the
intersection of the sets associated with $D_2$ and $D_4$ will not be empty
and element 1 of $D_5$ will be recorded as a constant.



}
