Static single assignment (SSA) form for scalar variables has been a significant
advance.  It has simplified the design of some optimizations and has made
other optimizations more effective.  
The popularity of SSA form surged
after an efficient algorithm for computing SSA form was made available~\cite{Cytron:1991:TOPLAS}.  SSA form is now a standard representation used in modern
optimizing compilers in both industry and academia.

However, it has been widely recognized that SSA form is much less
effective for array variables than for scalar variables.
The approach recommended in \cite{Cytron:1991:TOPLAS} is to treat an
entire array like a single scalar variable in SSA form.
\REM{
For example, in this approach an assignment to a single array element
$A[j] := v$ gets translated to an operation on the
entire array $A := Update(A, j, v)$ which after SSA renaming would
become $A_2 := Update(A_1, j, v)$.
}
The most serious limitation of this approach is that it lacks precise
data-flow information on a per-element basis.
Array SSA form addresses this limitation by providing $\Phi$
functions that can combine array values on a per-element basis.
The constant propagation algorithm described in this
chapter can  propagate lattice values through $\Phi$ functions 
in Array SSA form,
just
like any other operation/function in the input program.

The problem of conditional constant propagation for 
scalar variables 
%(with an interaction
%between constant propagation and unreachable code motion) 
has been
studied for several years.  Wegbreit~\cite{Wegb75} provided a general
algorithm for solving data-flow equations; his algorithm
can be used to perform conditional constant propagation and
more general combinations of program analyses.
However, his algorithm was too slow to be practical for use on 
large programs.
Wegman and Zadeck~\cite{WZ91} introduced a Sparse Conditional
Constant (SCC) propagation algorithm that is as precise as the conditional
constant propagation obtained by Wegbreit's algorithm, but runs faster
than Wegbreit's algorithm by a speedup factor that is at least $O(V)$,
where $V$ is the number of variables in the program.  
The improved efficiency of the SCC algorithm made it practical to
perform conditional constant propagation on large programs, even in the
context of industry-strength product compilers.  The main limitation
of the SCC algorithm is a conceptual one --- the algorithm operates
on two ``worklists''
(one containing edges in the SSA graph and another
containing edges from the control-flow graph)
rather than on data-flow equations.  The lack of
data-flow equations
makes it hard to combine the algorithm in \cite{WZ91}
with other program analyses.
The problem  of combining different program analyses based on
scalar SSA form has been addressed by Click and
Cooper in \cite{ClCo95}, where they present a framework for 
combining constant propagation, unreachable-code elimination,
and value numbering that explicitly uses data-flow equations.

Of the conditional constant propagation algorithms mentioned above,
our work is most closely related to that of \cite{ClCo95} with two 
significant differences.
First, our algorithm performs conditional constant propagation through
both scalar and array references, while the algorithm in \cite{ClCo95}
is limited only to scalar variables.
Second,  the framework in \cite{ClCo95}
uses control-flow predicates instead of execution flags.
It wasn't clear from the description in \cite{ClCo95} how their framework
deals with predicates that are logical combinations of multiple branch
conditions; it appears that
they must either allow the possibility of
an arbitrary size predicate expression
appearing in a data-flow equation (which would increase the worst-case
execution time complexity of their algorithm) or they must sacrifice
precision by working with an approximation of the predicate expression.


The analysis framework based on heap arrays reported in this chapter 
can be viewed as a flow-sensitive extension of
{\it type-based alias analysis} as in \cite{DiwanMM1998}.
Three different versions of type-based alias analysis were reported in
\cite{DiwanMM1998} --- {\it TypeDecl} (based on declared types of object
references), {\it FieldTypeDecl} (based on type declarations of
fields) and {\it SMTypeRefs} (based on an inspection of assignment
statements in the entire program).  All three versions are
flow-insensitive.  The disambiguation provided by heap arrays in our
approach is comparable to the disambiguation provided by {\it
FieldTypeDecl} analysis.  However, the use of value numbering and
Array SSA form in our approach results in flow-sensitive analyses of
array and object references that are more general than the three
versions of type-based alias analysis in \cite{DiwanMM1998}.  For
example, none of the three versions would disambiguate references {\tt
p.x} and {\tt q.x} in the example discussed earlier in
Figure~\ref{fig:ex2}(a).

Early algorithms for scalar replacement (\eg\
\cite{CaCK90}) were based on data dependence analysis and limited
their attention to loops with restricted control flow.
More recent algorithms for scalar replacement (\eg\
\cite{CaKe94,BoGu95}) use analyses based on PRE (partial redundancy
elimination) as an extension to data dependence analysis.  However,
all these past algorithms focused on accesses to elements of named
arrays, as in Fortran, and did not consider the possibility of
pointer-induced aliasing of arrays.  Hence, these algorithms are not
applicable to programming languages (such as C and Java) where arrays
might themselves be aliased.

Analysis and optimization of pointer references
in weakly typed languages such as C is a
major challenge because the underlying language semantics forces the
default assumptions to be highly conservative.  It is usually
necessary to perform a complex points-to analysis before pointer
references can be classified as {\it stack}-directed or {\it
heap-directed} and any effective optimization can be
performed~\cite{ghiya-hendren-98}.
To address this challenge,
there has been a large body of research on flow-sensitive
pointer-induced alias analysis in weakly typed languages \eg\
\cite{LaHi88,ChWZ90,HendrenHN1992,LandiRZ1993,ChBC93}.  
However,  these algorithms are too complex for use in efficient 
analysis of strongly typed languages, compared to the algorithms
presented in this chapter.
Specifically, our algorithms
analyze object references in the same SSA framework that has been
used in the past for efficient scalar analysis.  The fact that our
approach uses global value numbering in SSA form (rather than pointer
tracking) to determine if two pointers are the same or different leads
to significant improvements in time and space efficiency, because a
single value partition is used to supply information for all program
points rather than building different connection graphs at different
program points.
