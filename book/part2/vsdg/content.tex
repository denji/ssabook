\chapter{Gated SSA, Predicated SSA, Value State Dependence Graph \Author{J. Stanier}}
\numberofpages{15}
%\chapterauthor{J. Stanier, A. Mycroft}

%\TODO{including Gated SSA and Predicated SSA}
%\TODO{note overlap with other predicated sections}

\section*{Chapter outline}

\subsection*{Introduction}

This chapter will present work on three different SSA-based extensions: Gated Single Assignment, Predicated Static Single Assignment and the Value State Dependence Graph. Our explanation proceeds by explaining the difficulty of directly interpreting $\phi$-functions as there is no way to explicitly tell at compile-time which of the $\phi$-variables will be chosen. This introduces the idea of the gating function -- called the $\gamma$-node -- which was introduced by Ballance et al. in the Program Dependence Web\cite{207115} representation. We give some examples for explanation of $\phi$-translation, showing how it translates $\phi$-functions into $\gamma$-nodes.

\subsection*{Gated Single Assignment}

We then explore Havlak's work\cite{Havlak93constructionof} on Thinned Gated Single Assignment (TGSA), which is an extension of SSA with the aforementioned $\gamma$-nodes to represent a notion of control over value merges. We give some construction examples and show how loops are handled with $\mu$ and $\eta$ style representations. A short summary of the results of Havlak's implementation is given which highlights how this form is beneficial in symbolic analysis.

\subsection*{Predicated Static Single Assignment}

Predicated Static Single Assignment (PSSA), as introduced by Carter et al.\cite{carter99predicated} seeks to provide efficient optimizations for processors that have support for predicated execution. This means the processor has an additional set of predicate registers, which associate a guarding predicate with operations. This operation is only executed if the guarding predicate is true. We show how PSSA can be constructed with relation to the concept of hyperblocks, and then show how this helps in easing control restraints and instruction scheduling. Results of an implementation of PSSA are shown with conclusions.

\subsection*{Value State Dependence Graph}

The Value State Dependence Graph (VSDG) is an intermediate representation for modeling programs using data dependencies. It can be seen as a spiritual successor to the Value Dependence Graph\cite{177907}, with the addition of ``state" edges to allow only the essential sequential dependences in a program to constrain the ordering of instructions. The IR was introduced by Johnson\cite{UCAM-CL-TR-607,johnson-combined}, with further developments by Upton\cite{upton} and Lawrence\cite{UCAM-CL-TR-705}.

We give an outline of the key features of the VSDG and show how it is implicitly in SSA form. Construction is discussed with reference to Johnson's work. We then explain how the representation lends itself well to optimizations due to the adaptable nature of the graph. Discussion then focuses on the difficulties of generating sequential code from the graph\cite{DBLP:conf/pdpta/Upton03}, and Lawrence's proposed compiler architecture. We conclude with possible directions for ongoing work with the VSDG.
