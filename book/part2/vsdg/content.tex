\chapter{Graphs and gating functions \Author{J. Stanier}}
\numberofpages{15}

%\TODO{note overlap with other predicated sections}

\section*{Chapter outline}

\subsection*{Introduction}

This chapter will explore some approaches to using SSA concepts in graph-based intermediate representations. 

\subsection{Cliff Click's simple IR}

Click and Paleczny  \cite{202534} present a simple intermediate representation which is built from SSA form. This representation shows how SSA form can be transformed into a data flow graph, using $\phi$ nodes to represent data merges. Some simple examples are given that illustrate the main concepts, and results of the authors' implementation are given.

\subsection{The gating function}

One problem with SSA is the difficulty of directly interpreting $\phi$-functions as there is no way to explicitly tell at compile-time which of the $\phi$-variables will be chosen. This introduces the idea of the gating function -- called the $\gamma$-node -- which was introduced by Tu and Padua \cite{207115} and used by Ballance et al. in the Program Dependence Web\cite{93578} representation. We give some examples for explanation of $\phi$-translation, showing how it translates $\phi$-functions into $\gamma$-nodes.

\subsection*{Gated Single Assignment}

We then explore Havlak's work\cite{Havlak93constructionof} on Thinned Gated Single Assignment (TGSA), which is an extension of SSA with the aforementioned $\gamma$-nodes to represent a notion of control over value merges. We give some construction examples and show how loops are handled with $\mu$ and $\eta$ style representations. A short summary of the results of Havlak's implementation is given which highlights how this form is beneficial in symbolic analysis.

\subsection*{Value State Dependence Graph}

The Value State Dependence Graph (VSDG) is an intermediate representation for modeling programs using data dependencies. It can be seen as a spiritual successor to the Value Dependence Graph\cite{177907}, with the addition of ``state" edges to allow only the essential sequential dependences in a program to constrain the ordering of instructions. The IR was introduced by Johnson\cite{UCAM-CL-TR-607}, with further developments by Upton\cite{upton} and Lawrence\cite{UCAM-CL-TR-705}.

We give an outline of the key features of the VSDG and show how it is implicitly in SSA form. Construction is discussed with reference to Johnson's work. We then explain how the representation lends itself well to optimizations due to the adaptable nature of the graph. Discussion then focuses on the difficulties of generating sequential code from the graph\cite{DBLP:conf/pdpta/Upton03}, and Lawrence's proposed compiler architecture. We conclude with possible directions for ongoing work with the VSDG.

\subsection*{VSDG application: Combined Code Motion and Register Allocation}

Johnson and Mycroft \cite{johnson-combined} present an algorithm which performs code motion and register allocation at the same time by using the VSDG, uniting two phases which were traditionally believed to be antagonistic. We explore this algorithm and show how it works.
